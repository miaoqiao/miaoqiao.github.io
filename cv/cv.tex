%----------------------------------------------------------------------------------------
%	PACKAGES AND OTHER DOCUMENT CONFIGURATIONS
%----------------------------------------------------------------------------------------

\documentclass{resume} % Use the custom resume.cls style
\usepackage{hyperref}
\usepackage[left=0.6in,top=0.6in,right=0.7in,bottom=0.6in]{geometry} % Document margins
\usepackage[symbol*]{footmisc}
\usepackage{pdfpages}
\def\items{\itemsep +0pt}

\name{Miao Qiao} 
\address{Room 524, Science Centre 303, University of Auckland}
\address{38 Princes Street, Auckland 1010, New Zealand}
\address{Tel: (+64) 93737599-88941 \quad Email: miao.qiao@auckland.ac.nz}

\begin{document}

%----------------------------------------------------------------------------------------
%	EDUCATION SECTION
%----------------------------------------------------------------------------------------

\begin{rSection}{Education and Employment}
{\bf Senior Lecturer (above the bar)} \hfill {\em 2023 - present}\\
School of Computer Science  \\ 
University of Auckland, New Zealand \\ 
{\bf Senior Lecturer} equivalent to Associate Professor in Tenure Track systems \hfill {\em 2021 - 2022}\\
School of Computer Science  \\ 
University of Auckland, New Zealand \\
{\bf Lecturer} \hfill {\em 2018 - 2020}\\
School of Computer Science \\  	
University of Auckland, New Zealand \\
{\bf Lecturer} \hfill {\em 2016 - 2018}\\
School of Engineering and Advanced Technology \\ 
Massey University, New Zealand \\
{\bf Researcher} \hfill {\em 2015-2016} \\
Advanced Digital Sciences Center\\
Singapore\\
{\bf Postdoctoral Fellow} \hfill {\em 2013 - 2015}\\
Department of Computer Science and Engineering \\
The Chinese University of Hong Kong, Hong Kong \\
{\bf Doctor of Philosophy} \hfill {\em 2009 - 2013}\\
Systems Engineering and Engineering Management \\
The Chinese University of Hong Kong, Hong Kong \\
{\bf Bachelor of Engineering} \hfill {\em 2005 - 2009}\\
Computer Science and Engineering\\
Shanghai Jiao Tong University, China
\end{rSection}
\begin{rSection}{Funding}
\begin{itemize}
\item Principal Investigator of the project ``Advanced Graph Analytics for Human Brain Connectivity'', supported by New Zealand Singapore Data Science Programme, the Ministry of Business Innovation and Employment (MBIE) 
	New Zealand,  \textbf{NZD 3,000,000}.	\hfill {\em 2020 - 2024}
\item Principal Investigator of project ``Subgraph Matching: Theory and Practice'', supported by Marsden Fast-Start, Royal Society of New Zealand,  \textbf{NZD 300,000}.   \hfill {\em 2018 - 2024}
\end{itemize}
\end{rSection}
\begin{rSection}{Leadership and Services}
\textbf{Program Committee Chair:}
\begin{itemize}
	\item Australasian Database Conference (ADC)\hfill 2021
	\item Doctoral Consortium of International Semantic Web Conference (ISWC), Core rank A \hfill 2019
\end{itemize}
\textbf{Review Board Member: }
\begin{itemize}
	\item Proceedings of the ACM Symposium on Principles of Database Systems (PODS), Core rank A* \hfill 2023
	\item Proceedings of the Very Large Database Endowment (PVLDB), Core rank A* \hfill 2020-2021,2025
\end{itemize}
\textbf{Program Committee Member:}
\begin{itemize}
	\item International Joint Conferences on Artificial Intelligence (IJCAI), Core rank A* \hfill 2019-2021
	\item Association for the Advancement of Artificial Intelligence (AAAI), Core rank A* \hfill 2020
\end{itemize}
\textbf{Reviewer}:
\begin{itemize}
	\item Symposium on Computational Geometry, Core rank A,\hfill 2019
	\item Transactions on Database Systems (TODS), Core rank A*\hfill 2015, 2018, 2019
	\item Transactions on Knowledge and Data Engineering (TKDE), Core rank A*\hfill 2018
	\item International Conference on Knowledge Discovery \& Data Mining (KDD), Core rank A*\hfill 2016
	\item International Conference on Data Mining (ICDM), Core rank A*\hfill 2013, 2014
	\item Web Information Systems Engineering (WISE), Core rank A\hfill 2013
	\item International Journal on Very Large Data Bases (VLDBJ), Core rank A* \hfill 2013, 2018, 2019, 2024
	\item ACM Transactions on Database Systems (TODS), Core rank A* \hfill 2024
\end{itemize}
\textbf{Vice President} of South Pacific Competition Programming Association in 2023. \vspace{0.1cm}\\

\textbf{University Services:}
\begin{itemize}
\item \textbf{School-level} Stage 2 Coordinator \hfill 2025
\item \textbf{School-level} PhD coordinator (deputy) \hfill 2019-2020
\item \textbf{School-level} ACM International Collegiate Programming Contest coordinator \hfill 2018-2023 
\item \textbf{University-level} Member of the University of Auckland Human Participants Ethics Committee (UAHPEC) \hfill 2021-2023
\end{itemize}
\end{rSection}




% \begin{rSection}{Completed Workshops for PhD Supervision}
% \begin{itemize}
% 	\item Art of Graduate Research Supervision \hfill 2018, University of Auckland
% 	\item Orientation to Doctoral Education Policy and Process (ORIDOC) \hfill 2018, University of Auckland
% \end{itemize}
% \end{rSection}
% \vspace{3mm} 
\begin{rSection}{Talks}
% \begin{enumerate} \addtocounter{enumi}{21}
\begin{itemize}	
	\item 2024, Scalable Query Processing with Graphs. Data Systems Seminar Series, University of Waterloo
	\item 2024, Scalable Query Processing with Graphs. Database group, CMU
\end{itemize}
\end{rSection}
\begin{rSection}{Awards} 
\begin{itemize} 
	\item VLDB 2013 Travel Fellowship \hfill 2013
	\item Doctor Scholarship, The Chinese University of Hong Kong \hfill 2009-2013
	\item Shanghai Jiao Tong University 1st Class Scholarship \hfill 2008
	\item ACM International Collegiate Programming Contest (ICPC), 3rd Place, Singapore \hfill 2007
	\item Computer World Scholarship \hfill 2007
	\item Singapore Technology Engineering Scholarship \hfill 2006
	\item ACM International Collegiate Programming Contest (ICPC), 1st Place, Korea \hfill 2005
	\item Silver medal nationwide, National Olympiad in Informatics, China \hfill 2004
\end{itemize}
\end{rSection}

\newpage

\begin{rSection}{Teaching}
%I am comfortable teaching any courses related to database systems and big data algorithms. I am currently teaching three courses in both School of Computer Science and Software Engineering, University of Auckland: `Fundamentals of Database Systems' (COMPSCI 351 / SOFTENG 351) and `Advanced Topics in Database Systems' (COMPSCI751). I have received excellent teaching evaluations. For example, the satisfactory rate of the course COMPSCI751 2022 was 83.3\% and that of my teaching was 96.0\%. I also played a leadership role in developing the new course of ``COMPSCI 753 Algorithms for Massive Data'': after two years of consecutive effort, this postgraduate level course received a satisfaction rate of 80.6\% in 2019 and that of my teaching was 96.3\%. 
\begin{itemize} 
	\item COMPSCI751, Advanced Topics in Database Systems (2019-2023, 2025)
	\item SOFTENG351, Fundamentals of Database Systems (2019-2023, 2025)
	\item COMPSCI351, Fundamentals of Database Systems (2019-2023, 2025)
	\item COMPSCI753, Algorithms for Massive Data (2018-2019, 2025)
	\item 158.347, Database Paradigms (2017)
	\item 159.172, Computational Thinking and Algorithms (2017)
	\item 158.247, Database Design (2017-2018)
	\item 158.225, Systems Analysis and Modelling (2017-2018)
	\item 158.258, Web Development (2016)
	\item 158.172, Computational Thinking and Algorithms (2016)
	\item 158.225, Systems Analysis and Modelling (2016)
\end{itemize}

%I have supervised 7 PhD students, 9 master students and 10 honors students. 

Graduated Students:
\begin{itemize}
\item Wentao Li. PhD 2021. Lecturer at the University of Leicester since 2024. Wentao was selected as 2021 Global Top 100 Chinese Rising Stars in Artificial Intelligence. 
\item Zijin Feng. PhD 2024. Huawei, Hong Kong. 
\end{itemize}

Ongoing Students: 
\begin{itemize}
\item Yizhou Dai  
\item Callum Cory
% \item Yunhan Yang
\end{itemize}
\end{rSection}


% \begin{rSection}{Research Interests}
% My expertise lies in database research such as indexing, query optimization, graph analytics and its applications in brain networks. Specifically, my research covers three main categories: 
% \begin{itemize}
% 	\item Finding explainable and efficient algorithms for big data processing and analysis. The goal is to find big data algorithms that are empirically efficient with an efficiency theoretically explainable. Under this category, we have explored the computation of graph metrics, densest subgraph search, community detection, hypergraph clustering, I/O-efficient algorithm design and streaming algorithms. 
% 	\item Indexing and query optimization. The goal is to design data indexing to optimize online query performances. My past and ongoing research span over topics including graph distance queries, nearest neighbor search in high dimensional space, range thresholding queries on streams, multi-way join queries in traditional relational databases, local dense subgraph search, etc. 
% 	\item Brain network analysis. This interdisciplinary (with medical and health science) research aims at understanding and explaining the functions of the human brain, and bettering the predictions of the malfunctioning of the human brain with graph analytical techniques. We have created and are actively maintaining a large graph dataset constructed by transforming both publicly available and locally collected brain images (fMRI + T1) to graph data (we are planning to expand the data set to DTI data in the following year). The dataset is released to the public. 
% \end{itemize}

% My expertise lies at the intersection of database research, graph analytics, and interdisciplinary applications such as brain network analysis. Specifically, my research spans three primary categories:

% \begin{itemize} \item Explainable and Efficient Algorithms for Big Data Processing
% This line of work focuses on designing algorithms that are both empirically efficient and theoretically explainable, addressing the critical need for scalability in big data applications. My contributions include developing advanced methods for computing graph metrics, densest subgraph search, community detection, hypergraph clustering, and designing I/O-efficient and streaming algorithms. These efforts bridge the gap between practical utility and theoretical rigor, creating solutions tailored for real-world scalability challenges.
% \item **Indexing and Query Optimization**  
% This research aims to optimize online query performance through innovative data indexing and query processing strategies. My work has tackled a wide array of challenges, including efficient graph distance queries, Approximate Nearest Neighbor Search (ANNS) in high-dimensional spaces, range-thresholding queries on streams, multi-way join queries in relational databases, and local dense subgraph search. By combining foundational database principles with advanced graph analytics, these methods improve query efficiency and scalability in diverse data-intensive applications.  

% \item **Brain Network Analytics**  
% This interdisciplinary research, in collaboration with experts in medical and health sciences, focuses on understanding and explaining brain connectivity and improving predictions of neurological dysfunctions using graph-based analytical techniques. A key achievement has been the creation and ongoing maintenance of a comprehensive graph dataset derived from both publicly available and locally collected brain imaging data (fMRI and T1). This dataset is publicly available and serves as a resource for advancing brain network analysis. In the upcoming year, we plan to expand this dataset to include DTI data, enabling richer multi-modal integration for improved analysis of brain connectivity and its link to neurodegenerative conditions.  

% \end{rSection}


\begin{rSection}{Conference Publications}
%In the following, database theory papers are marked with `**'; in these publications, authors are ordered alphabetically, as is a convention in the theory field. Biomedical papers (e.g., in the Journal of Bioengineering) have a convention of listing the supervisors of the first author (usually a student) in the last of the author list. In the other papers, authors are ordered by contribution. The paper in which the main author is a student who completed the paper under the supervision of the applicant is marked with `*'.\\ 

\begin{enumerate}
% \begin{itemize}
	\item Yue Zeng, \textbf{Miao Qiao}, Rong-Hua Li, Hongchao Qin, Guoren Wang. Scaling Up k-Clique Percolation Community Detection. \emph{Proceedings of the ACM on Management of Data (SIGMOD)}, 2025.
	
	\item Zhencan Peng, \textbf{Miao Qiao}, Wenchao Zhou, Feifei Li, Dong Deng, Dynamic Range-Filtering Approximate Nearest Neighbor Search. \emph{VLDB}, 2025.

	\item *Zijin Feng, \textbf{Miao Qiao}, Chengzhi Piao, Hong Cheng. On Graph Representation for Attributed Hypergraph Clustering. \emph{Proceedings of the International Conference on Management of Data (SIGMOD)}, 2025. 

	\item Jiarui Luo, \textbf{Miao Qiao}, Chaoji Zuo, Dong Deng. Tag-Filtered Approximate Nearest Neighbor Search. \emph{IEEE International Conference on Data Engineering (ICDE)}, 2025. 


%	\item Jiaxing Xu, Kai He, Mengcheng Lan, Qingtian Bian, Wei Li, Tieying Li, Yiping Ke, \textbf{Miao Qiao}. Contrasformer: A Brain Network Contrastive Transformer for Neurodegenerative Condition Identification. \emph{Proceedings of the 33rd ACM International Conference on Information and Knowledge Management (CIKM)} 2024.
	
	\item Jiaxing Xu, Qingtian Bian, Xinhang Li, Aihu Zhang, Yiping Ke, \textbf{Miao Qiao}, Wei Zhang, Wei Khang Jeremy Sim, Balazs Gulyas. Contrastive Graph Pooling for Explainable Classification of Brain Networks \emph{IEEE TRANSACTIONS ON MEDICAL IMAGING (TMI) }, Vol 43, page 3292, 2024.

	\item Chaoji Zuo, \textbf{Miao Qiao}, Wenchao Zhou, Feifei Li, Dong Deng.
	SeRF: Segment Graph for Range-Filtering Approximate Nearest Neighbor Search. \emph{Proceedings of the International Conference on Management of Data (SIGMOD)} 2(1): 69:1-69:26 (2024)

	\item *Yizhou Dai, \textbf{Miao Qiao}, Rong-Hua Li. On Density-based Local Community Search. \emph{Proceedings of the 34th ACM Symposium on Principles of Database Systems (PODS)} 2(2): 88, 2024


	\item Jiaxing Xu, Kai He, Mengcheng Lan, Qingtian Bian, Wei Li, Tieying Li, Yiping Ke, \textbf{Miao Qiao}. Contrasformer: a brain network contrastive transformer for neurodegenerative condition identification. \emph{CIKM}, 2024.

	%\item Xiaowei Ye, \textbf{Miao Qiao}, Ronghua Li, Qi Zhang, Guoren Wang. Scalable $k$-clique Densest Subgraph Search. \emph{Under review, SIGMOD Round 4.}



	\item *Zijin Feng, \textbf{Miao Qiao}, Hong Cheng. Modularity-based Hypergraph Clustering: Random Hypergraph Model, Hyperedge-cluster Relation, and Computation. \emph{Proceedings of the International Conference on Management of Data (SIGMOD)}, 1(3): 215:1-215:25, 2024. 
	\item Jiaxing Xu, Yunhan Yang, David Tse Jung Huang, Sophi Shilpa Gururajapathy, Yiping Ke, \textbf{Miao Qiao}, Alan Wang, Haribalan Kumar, Josh McGeown, Eryn Kwon. Data-Driven Network Neuroscience: On Data Collection and Benchmark. \emph{Conference on Neural Information Processing Systems (NeurIPS)},  2023.
	\item Yiping Liu, Jiamou Liu, Bakh Khoussainov, \textbf{Miao Qiao}, Bo Yan, Mengxiao Zhang, Centralization Problem for Opinion Convergence in Decentralized Networks. \emph{Proceedings of the International Conference on Advances in Social Networks Analysis and Mining}, Pages 658-665, 2023. 
	\item *Yizhou Dai, \textbf{Miao Qiao}, Lijun Chang.	Anchored Densest Subgraph. \emph{Proceedings of the International Conference on Management of Data (SIGMOD)}, pages 1200-1213, 2022. 
	\item *Wentao Li, \textbf{Miao Qiao}, Lu Qin, Ying Zhang, Lijun Chang, Xuemin Lin. On Scalable Computation of Graph Eccentricities.	\emph{Proceedings of the International Conference on Management of Data (SIGMOD)}, pages 904-916, 2022. 

	\item *Zijin Feng, \textbf{Miao Qiao}, Hong Cheng. Clustering Activation Networks. 	\emph{Proceedings of the 34th International Conference on Data Engineering (ICDE)}, pages 780-792, 2022. 

	\item \textbf{Miao Qiao}, Yufei Tao.Two-Attribute Skew Free, Isolated CP Theorem, and Massively Parallel Joins. 	\emph{Proceedings of the 40th ACM Symposium on Principles of Database Systems (PODS)}, pages 166-180, 2021.  

	\item Hao Zhang, \textbf{Miao Qiao}, Jeffrey Xu Yu, Hong Cheng. Fast distributed complex join processing. \emph{2021 IEEE 37th International Conference on Data Engineering (ICDE)}. Pages 2087-2092, 2021.

	\item *Wentao Li, \textbf{Miao Qiao}, Lu Qin, Ying Zhang, Lijun Chang, Xuemin Lin. 	Scale Distance Labeling on Graphs with Core-Periphery Properties.   \emph{Proceedings of the International Conference on Management of Data (SIGMOD)}, pages 1367-1381, 2020. 


    \item Lijun Chang, \textbf{Miao Qiao}. 	Deconstruct Densest Subgraphs. \emph{Proceedings of the World Wide Web Conference (WWW)}, pages 2747-2753, 2020. 



	\item *Wentao Li, \textbf{Miao Qiao}, Lu Qin, Ying Zhang, Lijun Chang, Xuemin Lin. Scaling Distance Labeling on Small-World Networks  \emph{Proceedings of the International Conference on Management of Data (SIGMOD)}, pages 1060-1077, 2019. 


	\item *Wentao Li, \textbf{Miao Qiao}, Lu Qin, Ying Zhang, Lijun Chang, Xuemin Lin. Exacting Eccentricity for Small-World Networks. \emph{Proceedings of the 34th International Conference on Data Engineering (ICDE)}, pages 785-796, 2018. 


	\item \textbf{Miao Qiao},  Hao Zhang, Hong Cheng. Subgraph Matching: on Compression and Computation. \emph{Proceedings of the Very Large Database Endowment (PVLDB)}, 11(2): 176-188, 2017. 


	\item \textbf{Miao Qiao}, Junhao Gan, Yufei Tao. Range Thresholding on Streams.	\emph{Proceedings of the International Conference on Management of Data (SIGMOD)}, pages 571-582, 2016. 

	\item ** Xiaocheng Hu, \textbf{Miao Qiao}, Yufei Tao. Join Dependency Testing, Loomis-Whitney Join, and Triangle Enumeration. \emph{Proceedings of the 34th ACM Symposium on Principles of Database Systems (PODS)},  pages 291-301, 2015.  

	\item ** Xiaocheng Hu, \textbf{Miao Qiao}, Yufei Tao. External Memory Stream Sampling.\emph{Proceedings of the 34th ACM Symposium on Principles of Database Systems (PODS)}, pages 229-239, 2015. 

	
	\item ** Xiaocheng Hu, \textbf{Miao Qiao}, Yufei Tao. Independent Range Sampling.\emph{Proceedings of the 33rd ACM Symposium on Principles of Database Systems (PODS)}, pages 246-255, 2014. 
	
	
	\item \textbf{Miao Qiao}, Lu Qin, Hong Cheng, Jeffrey Xu Yu. Top-K Nearest Keyword Search on Large Graphs. \emph{Proceedings of the Very Large Database Endowment (PVLDB)}, 6(10): 901-912, 2013. 


	\item \textbf{Miao Qiao}, Hong Cheng, Lijun Chang, Jeffrey Xu Yu. Approximate shortest distance computing: a query-dependent local landmark scheme. \emph{Proceedings of the 28th International Conference on Data Engineering (ICDE)}, pages 462-473, 2012. 

	
	\item \textbf{Miao Qiao}, Hong Cheng, Jeffrey Xu Yu. Querying Shortest Path Distance with Bounded Errors in Large Graphs. 
	\emph{Proceedings of the 23rd International Conference on Scientific and Statistical Database Management (SSDBM)}, pages 255-273, 2011.
\end{enumerate}
% \end{itemize}	

\end{rSection}
\begin{rSection}{Journal Publications}
\begin{enumerate} 
	% \addtocounter{enumi}{21}
% \begin{itemize}
	\item Maryam Tayebi, Eryn Kwon, Josh McGeown, Leigh Potter, Davidson Taylor, Paul Condron, \textbf{Miao Qiao}, Patrick McHugh, Jerome Maller, Poul Nielsen, Alan Wang, Justin Fernandez, Miriam Scadeng, Vickie Shim, Samantha Holdsworth
	Characterizing the Effect of Repetitive Head Impact Exposure and mTBI on Adolescent Collision Sports Players’ Brain with Diffusion Magnetic Resonance Imaging. \emph{Journal of Neurotrauma}, 2025.
	
	\item Maryam Tayebi, Eryn Kwon, Jerome Maller, Josh McGeown, Miriam Scadeng, \textbf{Miao Qiao}, Alan Wang, Poul Nielsen, Justin Fernandez, Samantha Holdsworth, Vickie Shim, Matai mTBI Research Group Potter Leigh Condron Paul Taylor Davidson Cornfield Daniel McHugh Patrick Emsden Taylor Danesh-Meyer Helen Newburn Gil Bydder Graeme.  Integration of diffusion tensor imaging parameters with mesh morphing for in-depth analysis of brain white matter fibre tracts. \emph{Brain Communications}. 2024. 

	\item Jiaxing Xu, Qingtian Bian, Xinhang Li, Aihu Zhang,  Yiping Ke, \textbf{Miao Qiao}, Wei Zhang, Wei Khang Jeremy Sim, Balazs Gulyas. Contrastive Graph Pooling for Explainable Classification of Brain Networks. \emph{IEEE Transactions on Medical Imaging (TMI)}, 2024, \textbf{PREMIA Best Student Paper Honourable Mention Award}. \textbf{Impact factor $11.037$.} 

	\item *Wen Grace, Vickie Shim, Samantha Jane Holdsworth, Justin Fernandez, \textbf{Miao Qiao}, Nikola Kasabov, and Alan Wang. Machine Learning for Brain MRI Data Harmonisation: A Systematic Review, 2023. \emph{Bioengineering}, 10(4):397, 2023. \textbf{Impact factor $5.046$.}

	\item *Wentao Li, \textbf{Miao Qiao}, Lu Qin, Ying Zhang, Lijun Chang, Xuemin Lin. Distance labeling: on parallelism, compression, and ordering. \emph{Very Large Data Base Journal (VLDBJ)}, 31(1): 129-155, 2022. \textbf{Impact factor $4.243$.}  

	\item *Wentao Li, \textbf{Miao Qiao}, Lu Qin, Ying Zhang, Lijun Chang, Xuemin Lin.	Eccentricities on small-world networks. \emph{Very Large Data Base Journal (VLDBJ)}, 28(8): 1-28, 2019. \textbf{Impact factor $4.243$.} 

	\item Yufei Tao, Xiaocheng Hu, \textbf{Miao Qiao}. Stream Sampling over Windows with Worst-Case Optimality and $l$-Overlap Independence. \emph{Very Large Data Base Journal (VLDBJ)}, 26(4): 493-510, 2017.  \textbf{Impact factor $4.243$.} 

	\item ** Xiaocheng Hu, \textbf{Miao Qiao}, Yufei Tao. I/O-efficient join dependency testing, Loomis-Whitney join, and triangle enumeration. \emph{Journal of Computer and System Sciences} 82(8): 1300-1315 (2016)
	

	\item ** Xiaocheng Hu, \textbf{Miao Qiao}, Yufei Tao. Independent Range Sampling on a RAM. \emph{IEEE Data Engineering Bulletin} 38(3): 76-83 (2015)


	\item \textbf{Miao Qiao}, Hong Cheng, Lijun Chang and Jeffrey Xu Yu. Approximate Shortest Distance Computing: A Query-Dependent Local Landmark Scheme. \emph{IEEE Transactions on Knowledge and Data Engineering (TKDE)}, 26(1): 55-68, 2014. \textbf{Impact factor $7.05$.} 

	
	\item \textbf{Miao Qiao}, Hong Cheng, Lu Qin, Jeffrey Xu Yu, Philip S. Yu and Lijun Chang.  Computing Weight Constraint Reachability in Large Networks. \emph{International Journal on Very Large Data Bases (VLDBJ)}, 22(3): 275-294, 2012. \textbf{Impact factor $4.243$.} 

	
	\item Lijun Chang, Jeffrey Xu Yu, Lu Qin, Hong Cheng, \textbf{Miao Qiao}. The Exact Distance to Destination in Undirected World. \emph{International Journal on Very Large Data Bases (VLDBJ)}, 21(6): 869-888, 2012. \textbf{Impact factor $4.243$.} 
\end{enumerate}
% \end{itemize}
* denotes that the first author is a supervised PhD student; ** indicates equal contribution.
\end{rSection}


% \begin{rSection}{Talks}

% \begin{itemize}	
% 	\item 2024, Scalable Query Processing with Graphs. Data Systems Seminar Series, University of Waterloo
%    	\item 2024, Scalable Query Processing with Graphs. Database group, CMU
% \end{itemize}
% \end{rSection}


% \newpage
% \begin{rSection}{REFERENCES}
% \begin{tabular}{l l}
% % Dr. Jing Sun & Associate Professor \\
% % & School of Computer Science \\
% % & The University of Auckland \\
% % & Private Bag 92019, New Zealand\\
% % & Email: jing.sun@auckland.ac.nz \\
% Dr. Yufei Tao & Professor \hfill Postdoc supervisor\\
% & Department of Computer Science and Engineering  \\
% & The Chinese University of Hong Kong, Hong Kong \\
% & Email: taoyf@cse.cuhk.edu.hk \\
% Dr. Jeffrey Xu Yu & Professor \hfill PhD supervisor\\
% & Department of Systems Engineering and Engineering Management \\
% & The Chinese University of Hong Kong, Hong Kong \\
% & Email: yu@se.cuhk.edu.hk \\
% Dr. Hong Cheng & Professor \hfill PhD supervisor \\
% & Department of Systems Engineering and Engineering Management  \\
% & The Chinese University of Hong Kong, Hong Kong \\
% & Email: hcheng@se.cuhk.edu.hk \\
% Dr. Dong Deng & Assistant Professor \hfill Collaborator \\
% & Computer Science Department \\
% & Rutgers University \\
% & Email: dong.deng@rutgers.edu \\
% Dr. Lijun Chang & Associate Professor \hfill Collaborator \\
% & School of Computer Science \\
% & University of Sydney \\
% & Email: lijun.chang@sydney.edu.au\\
% \end{tabular}
% \end{rSection}



\end{document}
